
\documentclass[../main.tex]{subfiles}
\begin{document}

\section[Performance of PLLs]
{Part 2: Performance of PLLs}

\subsection{Tracking the Frequency of a Sinusoidal Signal}

\subsubsection{Transmitter VI Setup}
\hspace*{1.5em}{No narrative required for this section.}

\subsubsection{Receiver VI Setup}
\hspace*{1.5em}{No narrative required for this section.}

\subsubsection{Basic Measurements}
\begin{table}[H]
\centering
\begin{tabularx}{0.8\textwidth}
{|Y|Y|Y|Y|Y|Y|Y|}
\hline
% Labels
\textbf{Loop Type} & \textbf{$f_n$} & \textbf{$\Delta_f$} & 
\textbf{$\sigma$} & \textbf{$\zeta$} & \textbf{Freq. Pull-in} & 
\textbf{Phase Pull-in} \\
\hline\hline
% Data
Costas   & 2kHz  & 5kHz  & 0   & $1/\sqrt{2}$ & 1.25ms  & 1.7ms    \\ 
Costas   & 2kHz  & 10kHz & 0   & $1/\sqrt{2}$ & 6.33ms  & 6.76ms   \\
Standard & 5kHz  & 5kHz  & 0.1 & $1/\sqrt{2}$ & 61.63us & 198.57us \\
Standard & 10kHz & 5kHz  & 0.1 & $1/\sqrt{2}$ & 23.24us & 48.59us  \\
Standard & 1kHz  & 5kHz  & 0.1 & $1/\sqrt{2}$ & 3.19ms  & 3.99ms   \\
\hline
\end{tabularx}
\caption{Frequency and Phase Pull-in Times for Data Modulated Signals}
\end{table}

\begin{figure}[H]
   \centering
   \includegraphics[width=0.7\textwidth]{../lab_ss/p2_1_3_1_costas.png}
   \caption{2.1.3.1 - Costas PLL $f_n=2\text{ kHz},\Delta_f=5\text{ kHz},\sigma=0,\zeta=1/\sqrt{2}$}
\end{figure}
\begin{figure}[H]
   \centering
   \includegraphics[width=0.7\textwidth]{../lab_ss/p2_1_3_2_costas.png}
   \caption{2.1.3.2 - Costas PLL $f_n=2\text{ kHz},\Delta_f=10\text{ kHz},\sigma=0,\zeta=1/\sqrt{2}$}
\end{figure}
\begin{figure}[H]
   \centering
   \includegraphics[width=0.7\textwidth]{../lab_ss/p2_1_3_6.png}
   \caption{2.1.3.6 - Standard PLL $f_n=5\text{ kHz},\Delta_f=5\text{ kHz},\sigma=0.1,\zeta=1/\sqrt{2}$}
\end{figure}
\begin{figure}[H]
   \centering
   \includegraphics[width=0.7\textwidth]{../lab_ss/p2_1_3_7.png}
   \caption{2.1.3.7 - Standard PLL $f_n=10\text{ kHz},\Delta_f=5\text{ kHz},\sigma=0.1,\zeta=1/\sqrt{2}$}
\end{figure}
\begin{figure}[H]
   \centering
   \includegraphics[width=0.7\textwidth]{../lab_ss/p2_1_3_8.png}
   \caption{2.1.3.8 - Standard PLL $f_n=1\text{ kHz},\Delta_f=5\text{ kHz},\sigma=0.1,\zeta=1/\sqrt{2}$}
\end{figure}

\subsection{Tracking the Frequency of a Data Modulated Signal}
\subsubsection{Transmitter VI Setup}
\hspace*{1.5em}{No narrative required for this section.}
\subsubsection{Receiver VI Setup}
\hspace*{1.5em}{No narrative required for this section.}
\subsubsection{Measurements}
\begin{table}[H]
\centering
\begin{tabularx}{0.9\textwidth}
{|Y|Y|Y|Y|Y|Y|Y|Y|}
\hline
% Labels
\textbf{Loop Type} & \textbf{$f_n$} & \textbf{$\Delta_f$} & 
\textbf{$\sigma$} & \textbf{$\zeta$} & \textbf{$\mathbb{M}$} &
\textbf{Freq. Pull-in} & \textbf{Phase Pull-in} \\
\hline\hline
% Data
Standard & 0.5kHz  & 4kHz  & 0 & $1/\sqrt{2}$ & 1 & NoTrack  & NoTrack \\ 
Costas   & 0.5kHz  & 4kHz  & 0 & $1/\sqrt{2}$ & 1 & 62.66ms  & 65.2ms  \\
Standard & 0.5kHz  & 4kHz  & 0 & $1/\sqrt{2}$ & 2 & NoTrack  & NoTrack \\
Costas   & 0.5kHz  & 4kHz  & 0 & $1/\sqrt{2}$ & 2 & 59.3ms   & 67.3ms  \\
\hline
\end{tabularx}
\caption{Frequency and Phase Pull-in Times for Data Modulated Signals}
\end{table}

\begin{figure}[H]
   \centering
   \includegraphics[width=0.7\textwidth]{../lab_ss/p2_2_3_1.png}
   \caption{2.2.3.1 - Standard PLL $f_n=0.5\text{ kHz},\Delta_f=4\text{ kHz},\sigma=0,\zeta=1/\sqrt{2},\mathbb{M}=1$}
\end{figure}
\begin{figure}[H]
   \centering
   \includegraphics[width=0.7\textwidth]{../lab_ss/p2_2_3_2.png}
   \caption{2.2.3.2 - Costas PLL $f_n=0.5\text{ kHz},\Delta_f=4\text{ kHz},\sigma=0,\zeta=1/\sqrt{2},\mathbb{M}=1$}
\end{figure}
\begin{figure}[H]
   \centering
   \includegraphics[width=0.7\textwidth]{../lab_ss/p2_2_3_3.png}
   \caption{2.2.3.3 - Standard PLL $f_n=0.5\text{ kHz},\Delta_f=4\text{ kHz},\sigma=0,\zeta=1/\sqrt{2},\mathbb{M}=2$}
\end{figure}
\begin{figure}[H]
   \centering
   \includegraphics[width=0.7\textwidth]{../lab_ss/p2_2_3_4.png}
   \caption{2.2.3.4 - Costas PLL $f_n=0.5\text{ kHz},\Delta_f=4\text{ kHz},\sigma=0,\zeta=1/\sqrt{2},\mathbb{M}=2$}
\end{figure}

% Write-Up section
\subsection{Write-Up}
\subsubsection{} 
\textit{Using the data, discuss the relative performance of the PLLs with respect to the following
parameters:  
(a) frequency-pull-in time vs.\ phase-pull-in time versus damping \(\zeta\); \newline
(b) phase-pull-in time vs.\ total phase noise versus \(f_n\); \newline
(c) ability to lock in the presence of BPSK/QPSK data.}

\noindent\textbf{Answer.}  
(a) Light damping (\(\zeta<1/\!\sqrt2\)) pulls frequency in quickly but produces more
ringing in the phase response. Heavy damping smooths the phase but lengthens
the time needed to remove the frequency error. A critically damped loop sits
between these two cases and gives a reasonable compromise between speed and
overshoot. \newline
(b) Raising \(f_n\) widens the loop bandwidth, so both the frequency pull-in time
and the phase pull-in time decrease. The cost of this wider bandwidth is that
the loop passes more noise, so the locked phase shows more jitter and a higher
total phase noise level. \newline 
(c) A standard PLL does not lock when data flips the carrier phase each symbol
because the data transitions look like additional phase error to the loop. A
Costas loop multiplies the in-phase and quadrature components so that the data
term cancels on average and the carrier component can be tracked.

\subsubsection{} 
\textit{How would you create a PLL to track a general \(2M\)-PSK signal whose possible
phases are \(\phi_k=({2\pi}/{2M})k\)?}
\newline
\noindent\textbf{Answer.}  
Raise the received signal
\(r(t)=A e^{j(\omega_c t+\phi_k)}\) to the \(2M\)-th power:
\(r^{2M}(t)=A^{2M} e^{j 2M \omega_c t}\). The symbol phase disappears because
\(e^{j 2M \phi_k}=1\) for every allowed \(\phi_k\). Track this new carrier with
a standard PLL running at \(2M\omega_c\), then divide the recovered phase (or
the NCO count) by \(2M\) to obtain the original carrier phase. There is still a
fixed phase ambiguity that can be resolved with a differential decoder or a
short known preamble.


\end{document}