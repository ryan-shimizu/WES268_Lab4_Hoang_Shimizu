
\documentclass[../main.tex]{subfiles}
\begin{document}
\section[Phase Estimation Using PLL]
{Part 1: Phase Estimation Using PLL}
\subsection{Manual PLL}
\hspace*{1.5em}\textbf{1. - } 
{No narrative required for this section.}
\begin{figure}[H]
   \centering
   \includegraphics[width=0.7\textwidth]{../lab_ss/p1_1a.png}
   \caption{1.1 - Output of Phase Detector with Freq. Offset = 0Hz, Bandwidth = 2, Noise = 0}
\end{figure}

% Write-Up section
\subsection{Write-Up}
\subsubsection{}
\textit{Qualitatively explain the results of experiments using the manual PLL with
respect to the loop-filter bandwidth, the amount of noise, and the speed of the
loop's response.}


\noindent\textbf{Answer.}  
With a wide loop filter the VCO sees plenty of low-frequency gain, so it wipes
out frequency error almost immediately.  The downside is that the same wide
band lets noise into the loop, and even moderate noise (\(\sigma \approx 0.2\))
makes the phase jitter and can break lock.  A narrow filter quiets the noise
and keeps the phase steady, but the VCO now moves more slowly, so pull-in takes
longer, especially when the initial offset is large.  In short, wide bandwidth
is fast but jittery, while narrow bandwidth is slow but steady.


\end{document}