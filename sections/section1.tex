
\documentclass[../main.tex]{subfiles}
\begin{document}
\section[Channel Estimation using a PN Sequence]
{Part 1: Channel Estimation using a PN Sequence}
\subsection{Impulse Response Estimation of a Multipath Channel}
\subsubsection{Basic Measurements}
\hspace*{1.5em}\textbf{5. - }
{Insert}
\newline
\hspace*{1.5em}\textbf{6. - }
\begin{figure}[H]
   \centering
   \includegraphics[width=0.7\textwidth]{../lab_ss/p1_1_1_56.png}
   \caption{1.1.1.5/6 - Frequency at DC Component and PN sequence spectrum}
\end{figure}

\noindent\hspace*{1.5em}\textbf{7/8. - } 
\begin{figure}[H]
   \centering
   \includegraphics[width=0.7\textwidth]{../lab_ss/p1_1_1_78.png}
   \caption{1.1.1.7/8 - Period of Delta spikes of autocorrelation graph}
\end{figure}
\hspace*{1.5em}\textbf{9/10. - } 
\begin{figure}[H]
   \centering
   \includegraphics[width=0.7\textwidth]{../lab_ss/p1_1_1_10_freqlock.png}
   \caption{1.1.1.9/10 - Period of Delta spikes of autocorrelation graph}
\end{figure}
{Insert}

\subsubsection{Fading Measurements and Equalization}
\hspace*{1.5em}\textbf{2. - }
\begin{figure}[H]
   \centering
   \includegraphics[width=0.7\textwidth]{../lab_ss/p1_1_2_2.png}
   \caption{1.1.2.2 - Gains of Direct and Echo Path Terms}
\end{figure}
{Insert}
\newline
\hspace*{1.5em}\textbf{3. - } 
\begin{figure}[H]
   \centering
   \includegraphics[width=0.7\textwidth]{../lab_ss/p1_1_2_34_GOATED.png}
   \caption{1.1.2.3 - Gains of Direct and Echo Path Terms using fading paragrameters at receiver}
\end{figure}
\noindent\hspace*{1.5em}\textbf{4. - } 
{Insert}
\newline
\hspace*{1.5em}\textbf{5a. - } 
\begin{figure}[H]
   \centering
   \includegraphics[width=0.7\textwidth]{../lab_ss/p1_1_2_5a2.png}
   \caption{1.1.2.5a - Gains of Direct and Echo Path Terms, PN Order of 6}
\end{figure}
\begin{figure}[H]
   \centering
   \includegraphics[width=0.7\textwidth]{../lab_ss/p1_1_2_5a3.png}
   \caption{1.1.2.5a - Gains of Path Terms using transmitter fading parameters, PN Order of 6}
\end{figure}
\begin{figure}[H]
   \centering
   \includegraphics[width=0.7\textwidth]{../lab_ss/p1_1_2_5a4_check.png}
   \caption{1.1.2.5a - Gains of Path Terms using measured fading parameters, PN Order of 6}
\end{figure}
\noindent\hspace*{1.5em}\textbf{5b. - } 
\begin{figure}[H]
   \centering
   \includegraphics[width=0.7\textwidth]{../lab_ss/p1_1_2_5b2.png}
   \caption{1.1.2.5b - Gains of Direct and Echo Path Terms, PN Order of 7}
\end{figure}
\begin{figure}[H]
   \centering
   \includegraphics[width=0.7\textwidth]{../lab_ss/p1_1_2_5b34.png}
   \caption{1.1.2.5b - Gains of Path Terms using transmitter fading parameters, PN Order of 7}
\end{figure}
\noindent\hspace*{1.5em}\textbf{6. - } 
{Insert}
\newline
\hspace*{1.5em}\textbf{7. - } 
{Insert}
\newline
\hspace*{1.5em}\textbf{8. - } 
{Insert}
\newline

% Write-Up section
\subsection{Write-Up}
\subsubsection{}
\textit{Describe how you would create an FIR approximation of the equalization filter $g(n)$ denoted
g(n). Assuming the following H(z):
\newline
$$H(z) = 1 + Az^{-T}$$
$$G(z) = \frac{1}{1 + Az^{-T}}$$
You may assume that $|A| < 1$, and $T$ is a positive integer. Hint: What can you say about
$g(n)$ for large values of $n > N$.
}


\noindent\textbf{Answer.}  
Insert
\newline\newline
\textit{Now, for values $A = 0.5$ and $T = 4$, find a $\hat{g}(n)$.}
\newline
\noindent\textbf{Answer.}  
Insert

\subsubsection{}
\textit{
Based on your observations from this lab, describe the relationship between invertibility of
a channel $H(z)$ in respect to its poles and/or zero locations. In other words, what kind of
channels $h[n]$ are invertible (i.e. does a causal equalizer filter $g[n]$ exist?) and non-invertible?
}
\newline
\noindent\textbf{Answer.}  
Insert

\subsubsection{}
\textit{
a) Suppose that we wish to estimate the channel using a particular training sequence called the
"WES" sequence. All you know about this code is that it has very good auto-correlation
properties $ \Sigma w(k)w^{*}(n + k) = \delta(n)$ and the following quasi-periodic property:
$$w(n) = \begin{cases}
w(n+mL) & 0 \leq m \leq 9,\\
0  & \text{otherwise}
\end{cases}$$
Effectively, the "WES" sequence is a length $L$ sequence that repeats itself 10 times. Now,
assuming the input to your receiver is the "WES" sequence $w(n)$ with a frequency offset of
$f_0$
$$x(n) = w(n)e^{j2\pi f_0 n}$$
Find the cross-correlation between the received sequence $x(n)$ and the "WES" sequence
$w(n)$, denoted $r_{xw}(n)$
$$r_{xw}(n) = \sum_{k} x(n + k) w^{*}(k)$$
for values of $n = \{0, L, . . . , 9L\}$, you may assume that the term A is a constant where
$$A = \sum_{k} |w(k)|^2e^{j2\pi f_0 k}$$
}
\newline
\noindent\textbf{Answer.}  
Insert
\newline\newline


\noindent\textit{
b) Using your calculation of $r(0)$, and the fact that $|w(k)|^2 = 1$, comment on the effect that
frequency offset has on the magnitude of the cross-correlation.
}
\newline
\noindent\textbf{Answer.}  
Insert
\newline\newline

\noindent\textit{
c) Explain how you would calculate the frequency offset from your calculated values of
$r_{xw}(mL)$.
}
\newline
\noindent\textbf{Answer.}  
Insert
\newline\newline


\noindent\textit{
d) What is the maximum frequency offset $f_{max}$ that can be tracked using the cross-correlation approach and the "WES" code $w(n)$ of length $L$ (Hint: Think of the nyquist
sampling theorem).
}
\newline
\noindent\textbf{Answer.}  
Insert
\newline\newline

\end{document}